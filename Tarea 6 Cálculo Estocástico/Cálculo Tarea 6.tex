\documentclass[letterpaper]{article} 
\usepackage[left = 0.5in, right = 0.5in, top = 0.9in, bottom = 0.9in]{geometry}
\usepackage{enumitem}
\usepackage{multicol}
\usepackage[spanish]{babel}
\usepackage[utf8]{inputenc}

\usepackage{amsmath,amssymb,amsthm}
\usepackage{tikz-cd}
\usepackage{mathrsfs}
\usepackage[bbgreekl]{mathbbol}
\usepackage{dsfont}
\usepackage{graphicx}
\graphicspath{{img/}}

\newcommand{\op}{\operatorname}
\newcommand{\Op}{^{\op{op}}}
\newcommand{\scc}{\mathscr C}
\newcommand{\scd}{\mathscr D}
\newcommand{\sce}{\mathscr E}
\newcommand{\sci}{\mathscr I}
\newcommand{\scj}{\mathscr J}
\newcommand{\scx}{\mathscr X}
\newcommand{\var}{\mathrm{Var}}
\newcommand{\Id}{\operatorname{Id}}
\newcommand{\N}{\mathbb N}
\newcommand{\Z}{\mathbb Z}
\newcommand{\Q}{\mathbb{Q}}
\newcommand{\I}{\mathbb{I}}
\newcommand{\R}{\mathbb{R}}
\newcommand{\C}{\mathbb{C}}
\newcommand{\F}{\mathcal{F}}
\newcommand{\G}{\mathcal{G}}
\newcommand{\B}{\mathcal{B}}
\newcommand{\abs}[1]{\left\lvert #1 \right\rvert}
\newcommand{\inv}{^{-1}}
\renewcommand{\to}{\rightarrow}
\newcommand{\ent}{\Longrightarrow}
\newcommand{\E}{\mathbb{E}}
\renewcommand{\P}{\mathbb{P}}
\newcommand{\1}{\mathds{1}}
\renewcommand{\qedsymbol}{$\blacksquare$}

\theoremstyle{definition}
\newtheorem{dfn}{Definición}
\theoremstyle{definition}
\newtheorem{teo}{Teorema}
\theoremstyle{definition}
\newtheorem{cor}{Corolario}
\theoremstyle{definition}
\newtheorem{prop}{Proposición}
\theoremstyle{definition}
\newtheorem{obs}{Observación}


\title{\textbf{Cálculo Estocástico\\
Tarea 6}}
\author{Iván Irving Rosas Domínguez}
\date{\today}

\DeclareSymbolFontAlphabet{\mathbbm}{bbold}
\DeclareSymbolFontAlphabet{\mathbb}{AMSb}
\DeclareMathSymbol\bbDelta  \mathord{bbold}{"01}

\begin{document}
\maketitle

%\begin{abstract}
%\end{abstract}

\begin{enumerate}
    \item[\textbf{1.}] \textbf{Teorema (Existencia y unicidad para ecuaciones diferenciales estocásticas)}\\
    
    Sea $T>0$ y sean $b(\cdot, \cdot):[0,T]\times \R^{n}\to \R^{n}, \ \sigma(\cdot, \cdot):[0,T]\times \R^{n}\to \R^{n\times m}$ dos
    funciones medibles que satisfacen 
    \[
    |b(t,x)|+|\sigma(t,x)|\leq C(1+|x|), \qquad x\in \R^{n}, \  t\in [0,T]    
    \]
    para alguna constante $C>0$ (donde $|\sigma|^2=\sum|\sigma_{ij}|^2)$ y tales que 
    \[
    |b(t,x)-b(t,y)|+|\sigma(t,x)-\sigma(t,y)|\leq D|x-y|, \qquad x,y \in \R^{n}, \ t\in[0,T]    
    \]
    para alguna constante $D$. Sea $Z$ una variable aleatoria que es independiente de la $\sigma-$álgebra 
    $\F_\infty^{(m)}$ generada por $B_s(\cdot)$, $s\geq0$, y tal que 
    \[
        \E\left[|Z|^{2}\right]<\infty.
    \]
    Entonces la ecuación diferencial estocástica 
    \[
    dX_t=b(t,X_t)dt+\sigma(t,X_t)dB_t, \qquad 0\leq t\leq T, \ X_0=Z
    \]
    tiene una única solución $X_t(\omega)$, continua en la variable $t$ con la propiedad
    de que $X_t(\omega)$ es adaptada a la filtración $\F_t^{Z}$ generada por $Z$ y 
    $B_s(\cdot)$, con $s\leq t$ y 
    \[
    \E\left[\int_{0}^{T}|X_t|^{2}dt\right]<\infty.
    \]
    \item[\textbf{2.}] Verificar que los procesos dados resuelven las correspondientes ecuaciones diferenciales estocásticas:
     ($B_t$ denota el movimiento Browniano 1-dimensional)
     \begin{enumerate}
        \item $X_t=e^{B_t}$ resuelve $dX_t=\frac{1}{2}X_tdt+X_tdB_t$
        \item $X_t=\frac{B_t}{1+B_t}; \ B_0=0$ resuelve 
        \[
        dX_t=-\frac{1}{1+t}X_tdt+\frac{1}{1+t}dB_t; \qquad X_0=0.    
        \]
        \item $X_t=\sen(B_t)$ con $B_0=a \in \left(-\frac{\pi}{2},\frac{\pi}{2}\right)$ resuelve
        \[
            dX_t=-\frac{1}{2}X_t dt + \sqrt{1-X_t^2}dB_t \ \text{ para } \ t<T(\omega)=\inf \left\{s>0 : B_s \not \in \left[-\frac{\pi}{2},\frac{\pi}{2}\right]\right\}.
        \]
        \item $(X_1(t),X_2(t))=(t,e^{t}B_t)$ resuelve 
        \[
        \begin{bmatrix}
            dX_1\\
            dX_2
        \end{bmatrix}
        =
        \begin{bmatrix}
            1\\
            X_2
        \end{bmatrix}dt
        +
        \begin{bmatrix}
            0\\
            e^{X_1}
        \end{bmatrix}dB_t
        \]  
        \item $(X_1(t), X_2(t))=(\cosh(B_t),\sinh(B_t))$ resuelve 
            \[
            \begin{bmatrix}
                dX_1\\
                dX_2
            \end{bmatrix}
            =\frac{1}{2}
            \begin{bmatrix}
                X_1\\
                X_2
            \end{bmatrix}dt
            +
            \begin{bmatrix}
                X_2\\
                X_1
            \end{bmatrix}dB_t
            \]  
     \end{enumerate}
    \item[\textbf{3.}] Sea $(B_1,...,B_n)$ un movimiento Browniano en $\R^n$, $\alpha_1,...,\alpha_n$ 
    constantes. Resolver la ecuación diferencial estocástica 
    \[
    dX_y=rX_tdt+X_t \left(\sum_{k=1}^n \alpha_k dB_k(t)\right); \qquad X_0>0.    
    \]
    (Este es un modelo para un crecimiento exponencial con varias fuentes de ruido blanco independientes 
    en la tasa de crecimiento relativa.)
    \item[\textbf{4.}] Resolver las siguientes ecuaciones diferenciales estocásticas 
    \begin{enumerate}
        \item \[
            \begin{bmatrix}
            dX_1\\
            dX_2
        \end{bmatrix}= \begin{bmatrix}
            1\\
            0
        \end{bmatrix}dt + \begin{bmatrix}
            1 & 0\\
            0 & X_1
        \end{bmatrix} \begin{bmatrix}
            dB_1\\
            dB_2
        \end{bmatrix}.
        \]
        \item $dX_t=X_tdt + dB_t$. (Sugerencia: multiplicar ambos lados con el `factor integrante' $e^{-t}$
        y comparar con $d(e^{-t}X_t))$.
        \item $dX_t=-X_tdt+e^{-t}dB_t$.
    \end{enumerate}
    \item[\textbf{5.}] Resolver la ecuación diferencial estocástica (2-dimensional)
    \begin{align*}
        &dX_1(t)=X_2(t)dt+\alpha dB_1(t)\\
        &dX_2(t)=X_1(t)d+\beta dB_2(t)
    \end{align*}
    donde $(B_1(t),B_2(t))$ es un movimiento Browniano 2-dimensional y $\alpha$, $\beta$ son constantes.
    Este es un modelo para una cuerda vibrante sujeta a una fuerza estocástica.
    \item[\textbf{6.}] \textbf{(El puente Browniano).}
    
    Para $a,b\in \R$ fijos, considérese la siguiente ecuación $1$-dimensional.
    \[
    dY_t=\frac{b-Y_t}{1-t}dt+dB_t; \qquad 0\leq t<1, \quad Y_0=a.    
    \]
    Verificar que 
    \[
    Y_t=a(1-t) + bt + (1-t)\int_{0}^{t}\frac{dB_s}{1-s}; \qquad 0\leq t <1    
    \]
    resuelve la ecuación y demostrar que $\displaystyle \lim_{t\to 1}Y_t=b$ c.s. El proceso 
    $Y_t$ es denominado \textit{el puente Browniano} (de $a$ a $b$).
\end{enumerate}
\end{document}