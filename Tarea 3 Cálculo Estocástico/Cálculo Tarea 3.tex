\documentclass[letterpaper]{article} 
\usepackage[left = 0.5in, right = 0.5in, top = 0.9in, bottom = 0.9in]{geometry}
\usepackage{enumitem}
\usepackage{multicol}
\usepackage[spanish]{babel}
\usepackage[utf8]{inputenc}

\usepackage{amsmath,amssymb,amsthm}
\usepackage{tikz-cd}
\usepackage{mathrsfs}
\usepackage[bbgreekl]{mathbbol}
\usepackage{dsfont}
\newcommand{\op}{\operatorname}
\newcommand{\Op}{^{\op{op}}}
\newcommand{\scc}{\mathscr C}
\newcommand{\scd}{\mathscr D}
\newcommand{\sce}{\mathscr E}
\newcommand{\sci}{\mathscr I}
\newcommand{\scj}{\mathscr J}
\newcommand{\scx}{\mathscr X}
\newcommand{\var}{\mathrm{Var}}
\newcommand{\Id}{\operatorname{Id}}
\newcommand{\N}{\mathbb N}
\newcommand{\Z}{\mathbb Z}
\newcommand{\Q}{\mathbb{Q}}
\newcommand{\I}{\mathbb{I}}
\newcommand{\R}{\mathbb{R}}
\newcommand{\C}{\mathbb{C}}
\newcommand{\F}{\mathcal{F}}
\newcommand{\G}{\mathcal{G}}
\newcommand{\B}{\mathcal{B}}
\newcommand{\abs}[1]{\left\lvert #1 \right\rvert}
\newcommand{\inv}{^{-1}}
\renewcommand{\to}{\rightarrow}
\newcommand{\ent}{\Longrightarrow}
\newcommand{\E}{\mathbb{E}}
\renewcommand{\P}{\mathbb{P}}
\newcommand{\1}{\mathds{1}}
\renewcommand{\qedsymbol}{$\blacksquare$}

\theoremstyle{definition}
\newtheorem{dfn}{Definición}
\theoremstyle{definition}
\newtheorem{teo}{Teorema}
\theoremstyle{definition}
\newtheorem{cor}{Corolario}
\theoremstyle{definition}
\newtheorem{prop}{Proposición}
\theoremstyle{definition}
\newtheorem{obs}{Observación}


\title{\textbf{Cálculo Estocástico\\
Tarea 3}}
\author{Iván Irving Rosas Domínguez}
\date{\today}

\DeclareSymbolFontAlphabet{\mathbbm}{bbold}
\DeclareSymbolFontAlphabet{\mathbb}{AMSb}
\DeclareMathSymbol\bbDelta  \mathord{bbold}{"01}

\begin{document}
\maketitle

%\begin{abstract}
%\end{abstract}
\begin{enumerate}
    \item[\textbf{1. Lema:}] Sea $M^2$ el espacio de las martingalas cuadrado-integrables. Dada una martingala $X\in M^2$, definimos la métrica $\|.\|$ como sigue: 
    \[
        \|X\|:=\sum_{n=1}^{\infty} \frac{\|X\|_n\wedge1}{2^n},
    \]
    donde $\forall t\geq0$, $\|X\|_t:=\left(\E\left[X_t^2\right]\right)^{\frac{1}{2}}$.\\
    
    Entonces el espacio $\left((M^2,\|.\|)\right)$ es un espacio normado completo y el subespacio $M^2_c$ (el espacio de las martingalas cuadrado integrables continuas)
    es un subespacio cerrado de $M^2$.\\

    \item[\textbf{2.}] Muestra que si $X(t)$ es no aleatorio (no depende de $B(t)$) y es 
    función de $t$ y $s$ con $\int_{0}^{t}X^2(t,s)ds<\infty$, entonces $\int_{0}^{t}X(t,s)dB(s)$
    es una variable aleatoria Gaussiana $Y(t)$. La colección $Y(t)$, $0\leq t \leq T$, es
    un proceso Gaussiano con media cero y función de covarianza para $u\geq0$ dada por 
    $\text{Cov}\left(Y(t),Y(t+u)\right)=\int_{0}^{\infty}X(t,s)X(t+u,s)ds.$\\

    \item[\textbf{3.}] Muestra que una martingala Gaussiana en un intervalo de tiempo
    finito $[0,T]$ es una martingala cuadrado-integrable con incrementos independientes.
    Deducir que si $X$ es no aleatorio y $\int_{0}^{t}X^2(s)ds<\infty$, entonces $Y(t)=
    \int_{0}^{\infty}X(s)dB(s)$ es una martingala Gaussiana cuadrado-integrable con 
    incrementos independientes.\\

    \item[\textbf{4.}] Un proceso $X(t)$ en $(0,1)$ tiene un diferencial estocástico 
    con coeficiente $\sigma^2(x)=x(1-x)$. Suponiendo que $0<X(t)<1$, muestra que 
    el proceso definido por $Y(t)=\ln(X(t))/(1-X(t))$ tiene un coeficiente de 
    difusión constante.\\

    \item[\textbf{5.}] Obtener el diferencial de una fórmula de cociente $d \left(\frac{X(t)}{Y(t)}\right)$ 
    tomando $f(x,y)=x/y$. Suponga que el proceso $Y$ se mantiene lejos de 0.
\end{enumerate}
\end{document}