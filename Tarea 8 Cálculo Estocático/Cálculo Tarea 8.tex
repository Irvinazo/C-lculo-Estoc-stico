\documentclass[letterpaper]{article} 
\usepackage[left = 0.5in, right = 0.5in, top = 0.9in, bottom = 0.9in]{geometry}
\usepackage{enumitem}
\usepackage{multicol}
\usepackage[spanish]{babel}
\usepackage[utf8]{inputenc}

\usepackage{amsmath,amssymb,amsthm}
\usepackage{tikz-cd}
\usepackage{mathrsfs}
\usepackage[bbgreekl]{mathbbol}
\usepackage{dsfont}
\usepackage{graphicx}
\graphicspath{{img/}}

\newcommand{\op}{\operatorname}
\newcommand{\Op}{^{\op{op}}}
\newcommand{\scc}{\mathscr C}
\newcommand{\scd}{\mathscr D}
\newcommand{\sce}{\mathscr E}
\newcommand{\sci}{\mathscr I}
\newcommand{\scj}{\mathscr J}
\newcommand{\scx}{\mathscr X}
\newcommand{\var}{\mathrm{Var}}
\newcommand{\Id}{\operatorname{Id}}
\newcommand{\N}{\mathbb N}
\newcommand{\Z}{\mathbb Z}
\newcommand{\Q}{\mathbb{Q}}
\newcommand{\I}{\mathbb{I}}
\newcommand{\R}{\mathbb{R}}
\newcommand{\C}{\mathbb{C}}
\newcommand{\F}{\mathcal{F}}
\newcommand{\G}{\mathcal{G}}
\newcommand{\B}{\mathcal{B}}
\newcommand{\abs}[1]{\left\lvert #1 \right\rvert}
\newcommand{\inv}{^{-1}}
\renewcommand{\to}{\rightarrow}
\newcommand{\ent}{\Longrightarrow}
\newcommand{\E}{\mathbb{E}}
\renewcommand{\P}{\mathbb{P}}
\newcommand{\1}{\mathds{1}}
\renewcommand{\qedsymbol}{$\blacksquare$}

\theoremstyle{definition}
\newtheorem{dfn}{Definición}
\theoremstyle{definition}
\newtheorem{teo}{Teorema}
\theoremstyle{definition}
\newtheorem{cor}{Corolario}
\theoremstyle{definition}
\newtheorem{prop}{Proposición}
\theoremstyle{definition}
\newtheorem{obs}{Observación}


\title{\textbf{Cálculo Estocástico\\
Tarea 8}}
\author{Iván Irving Rosas Domínguez}
\date{\today}

\DeclareSymbolFontAlphabet{\mathbbm}{bbold}
\DeclareSymbolFontAlphabet{\mathbb}{AMSb}
\DeclareMathSymbol\bbDelta  \mathord{bbold}{"01}

\begin{document}
\maketitle

%\begin{abstract}
%\end{abstract}

\begin{itemize}
    \item[\textbf{1.}] Mostrar que para cualquier $u$, $f(x,t)=\exp \left\{ux-u^2t/2\right\}$ resuleve 
    la ecuación backward para el movimiento Browniano. Tómense las derivadas primeras, segundas, etc.,
    de $\exp \left\{ux-u^2t/2\right\}$ con respecto a $u$ e igualese $u=0$ para obtener que 
    las funciones $x$, $x^2-t$, $x^3-3tx$, $x^{4}-6tx^2+3t^2$, etc. también resuelven la ecuación backward 6.13.
    Deducir que $B^2(t)-t,B^3(t)-3tB(t)$ y $B^4(t)-6tB^2(t)+3t^2$ son martingalas.

    \item[\textbf{2.}] Hallar $f(x)$ tal que $f(B(t)+t)$ es una martingala.
    
    \item[\textbf{3.}] Investigar las explosiones del siguiente proceso 
    \[
        dX(t)=X^2(t)dt+\sigma X^{\alpha}(t)dB(t).  
    \]
    \item[\textbf{4.}] Mostrar que el cuadrado del proceso de Bessel $X(t)$ en 6.64 se acerca
    arbitrariamente al cero cuando $n=2$, esto es, $\P\left(T_y<\infty\right)=1$ para cualquier 
    $y>0$ suficientemente pequeña, pero cuando $n\geq3$, $\P\left(T_y<\infty\right)<1$.

    \item[\textbf{5.}] Sea $X(t)$ una difusión con coeficientes $\sigma(x)=1$, $\mu(x)=-1$ para $x<0$, 
    $\mu(x)=1$ para $x>0$ y $\mu(0)=0$. Mostrar que $\pi(x)=e^{-|x|}$ es una distribución 
    estacionaria para $X$.
\end{itemize}
\end{document}